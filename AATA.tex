\documentclass[12pt,a4paper]{article}
\usepackage{amssymb}
\usepackage{amsthm}
\usepackage{fontspec, xunicode, xltxtra}
\usepackage[slantfont, boldfont]{xeCJK}
\usepackage{titlesec}
\usepackage{indentfirst}
\usepackage{enumerate}
\usepackage{multicol}
\usepackage{multirow}
\usepackage[fleqn]{amsmath}
\usepackage{xcolor}
\usepackage{clrscode3e}
\usepackage{listings}
\usepackage{graphicx}
\usepackage{fancyhdr}
%\usepackage{algorithm}
%\usepackage{algorithmic}
%\usepackage{amsmath}
%\usepackage{sectsty}
%\usepackage{picinpar}
%\usepackage{bm}
%\usepackage[boxed,linesnumbered]{algorithm2e}


\setCJKmainfont[BoldFont={Adobe Heiti Std}, ItalicFont={Adobe Kaiti Std}]{Adobe Song Std}
\setCJKmonofont{Adobe Fangsong Std}
\setmainfont[Mapping=tex-text]{Liberation Serif}
\setsansfont{Liberation Sans}
\setmonofont{Bitstream Vera Sans Mono}
\punctstyle{kaiming}
\pagestyle{fancy}
\fancyhead{}
\fancyfoot{} % clear all fields
\fancyfoot[LF]{Copyright by Yinyanghu}
\fancyfoot[RF]{\thepage}
\renewcommand{\footrulewidth}{0.4pt}
\renewcommand{\headrulewidth}{0.0pt}
%\setCJKfamilyfont{kai}{KaiTi}
%\setCJKfamilyfont{hei}{SimHei}
%\setCJKmainfont{SimSun}


%\newcommand{\kai}{\CJKfamily{kai}}
%\newcommand{\hei}{\CJKfamily{hei}}
%\newcommand\TT{\rule{0pt}{2.6ex}}
%\newcommand\BB{\rule[-1.2ex]{0pt}{0pt}}



\newcommand\abs[1]{\left\lvert #1 \right\rvert}
\newcommand\floor[1]{\left\lfloor #1 \right\rfloor}
\newcommand\ceil[1]{\left\lceil #1 \right\rceil}
\setlength{\parindent}{2.5em}

\renewcommand{\labelenumi}{\bfseries{\alph{enumi}.}}
\renewcommand{\labelenumii}{\bfseries{\arabic{enumii}.}}




\begin{document}

\title{The solutions to the book \\ ``Abstract Algebra Theory and Applications'' \\ by Thomas W. Judson}
\author{Jian Li \\ Computer Science \\ Nanjing University, China}
\date{2011}
\maketitle
\pagebreak

\section{Preliminaries}
\pagebreak


\section{The Integers}

\subsection*{Problem 16}
\noindent Suppose $a$ and $b$ are not relatively prime, there exist a $k$ such that $k | a$, $k | b$ and $k \neq 1$. Therefore, we have
\begin{equation} \notag
	ar + bs = kpr + kqs = k(pr + qs) = 1
\end{equation}
So we have $k | 1$. It is controdict $k \neq 1$. So $a$ and $b$ are relateively prime.


\subsection*{Problem 28}
\noindent This problem equivalent the problem "Let $p \geq 2$, if $p$ is not prime, so is $2^p - 1$" \\
Since $p$ is not prime, there exist $k \neq 1$ such that $k | p$. Now we can think that $2^p - 1 = (111...1)_2$, a $p$-bit binary number. Therefore, we have $2^k - 1 | 2^p - 1$, since $(\underbrace{111...1}_\text{$k$-bits}) | (\underbrace{111...1}_\text{$p$-bits})$. So $2^p - 1$ is not a prime.



\pagebreak


\section{Groups}

\subsection*{Problem 24}
\noindent Proof:
\begin{equation} \notag
	(aba^{-1})^n = aba^{-1} aba^{-1} ... aba^{-1} = ab(a^{-1}a)b(a^{-1}a)b...(a^{-1}a)ba^{-1} = ab^na^{-1}
\end{equation}


\subsection*{Problem 30}
\noindent Since $a^2 = e$ for all $a \in G$, we have $a = a^{-1}$ for all $a \in G$. For any $a, b \in G$,
\begin{equation} \notag
	ab = a^{-1}b^{-1} = (ba)^{-1} = ba
\end{equation}
Therefore, $G$ is an abelian group.



\pagebreak


\section{Cyclic Groups}

\subsection*{Problem 29}
\noindent Since the number of generators of $\mathbf{Z}_n$ is $\phi(n)$, and $\phi(nm) = \phi(n) \phi(m)$, we have $\phi(m^k) = m^k - m^{k - 1}$ for any prime $k$. Therefore, $\phi(n)$ is even for all $n \geq 3$ and $\mathbf{Z}_n$ has an even number of generators for $n > 2$. 


\subsection*{Problem 30}
\noindent Suppose that there exist $p < m$, $q < n$ such that $a^p = b^q \neq e$, we have $b^{qm} = a^{pm} = e = b^n$. Since $\abs{b} = n$, we conclude that $n | qm$. And we can also conclude that $m | pn$ in the same way. It controdicts $gcd($m$, $n$) = 1$. Thus, $\langle a \rangle \cup \langle b \rangle = \{e\}$.


\subsection*{Problem 36}
\noindent Since $\mathbf{Z}_n$ is a cyclic group of order $n$ and 1 is a generator of the group. According to Theorem 4.6, we have the order of $r$ is $n / gcd(r, n)$. Since $r$ is a generator, therefore $gcd(r, n) = 1$. 



\pagebreak


\section{Permutation Group}

\subsection*{Problem 27}
\noindent One-to-One: \\
If $\lambda_g(a) = \lambda_g(b)$, we have $\lambda_g(a) = ga = gb = \lambda_g(b)$, which means $a = b$. \\
Onto: \\
For any $b \in G$, we can find that $\lambda_g(g^{-1}b) = gg^{-1}b = b$ and $g^{-1}b \in G$. \\

\noindent Therefore, $\lambda_g$ is a permutation of $G$.


\subsection*{Problem 31}
\noindent Reflexive: \\
Since $e \in S_n$, we know that $e \alpha e^{-1} = \alpha$ for all $\alpha \in S_n$. Thus, we have $\alpha \sim \alpha$. \\
Symmetric: \\
If there exist $\alpha \sim \beta$, it means that $\sigma \alpha \sigma^{-1} = \beta$ for some $\sigma \in S_n$. Then, we have $\alpha = \sigma^{-1} \beta \sigma = \sigma^{-1} \beta (\sigma^{-1})^{-1}$. Thus, we conclude that $\beta \sim \alpha$. \\
Transitive: \\
Suppose that we know that $\alpha \sim \beta$ and $\beta \sim \gamma$, it means that $\beta = \sigma \alpha \sigma^{-1}$ and $\gamma = \delta \beta \delta^{-1}$ for some $\sigma, \delta \in S_n$.
So $\gamma = \delta \sigma \alpha \sigma^{-1} \delta^{-1} = (\delta \sigma) \alpha (\sigma^{-1} \delta^{-1}) = (\delta \sigma) \alpha (\delta \sigma)^{-1}$ for some $\delta \sigma \in S_n$. \\

\noindent Therefore, $\sim$ is an equivalence relation on $S_n$.



\pagebreak

\section{Cosets and Lagrange's Theorem}

\subsection*{Problem 17}
\noindent if $a \notin H$, then $a^{-1} \notin H$ $\Rightarrow$ $a^{-1} \in aH = a^{-1}H = bH$ $\Rightarrow$ $\exists h_1, h_2 \in H$ s.t. $a^{-1}h_1 = bh_2$ $\Rightarrow$ $ab = h_1{h_2}^{-1} \in H$


\subsection*{Problem 18}
\noindent if $g \in H$, then $gH = Hg = H$; if $g \notin H$, then $gH = Hg = G - H$



\pagebreak


\section{Introduction to Cryptography}

\pagebreak

\end{document}

